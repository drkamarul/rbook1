\documentclass[]{book}
\usepackage{lmodern}
\usepackage{amssymb,amsmath}
\usepackage{ifxetex,ifluatex}
\usepackage{fixltx2e} % provides \textsubscript
\ifnum 0\ifxetex 1\fi\ifluatex 1\fi=0 % if pdftex
  \usepackage[T1]{fontenc}
  \usepackage[utf8]{inputenc}
\else % if luatex or xelatex
  \ifxetex
    \usepackage{mathspec}
  \else
    \usepackage{fontspec}
  \fi
  \defaultfontfeatures{Ligatures=TeX,Scale=MatchLowercase}
\fi
% use upquote if available, for straight quotes in verbatim environments
\IfFileExists{upquote.sty}{\usepackage{upquote}}{}
% use microtype if available
\IfFileExists{microtype.sty}{%
\usepackage{microtype}
\UseMicrotypeSet[protrusion]{basicmath} % disable protrusion for tt fonts
}{}
\usepackage[margin=1in]{geometry}
\usepackage{hyperref}
\hypersetup{unicode=true,
            pdftitle={Exploring data using R},
            pdfauthor={Kamarul Imran Musa, Wan Nor Arifin},
            pdfborder={0 0 0},
            breaklinks=true}
\urlstyle{same}  % don't use monospace font for urls
\usepackage{longtable,booktabs}
\usepackage{graphicx,grffile}
\makeatletter
\def\maxwidth{\ifdim\Gin@nat@width>\linewidth\linewidth\else\Gin@nat@width\fi}
\def\maxheight{\ifdim\Gin@nat@height>\textheight\textheight\else\Gin@nat@height\fi}
\makeatother
% Scale images if necessary, so that they will not overflow the page
% margins by default, and it is still possible to overwrite the defaults
% using explicit options in \includegraphics[width, height, ...]{}
\setkeys{Gin}{width=\maxwidth,height=\maxheight,keepaspectratio}
\IfFileExists{parskip.sty}{%
\usepackage{parskip}
}{% else
\setlength{\parindent}{0pt}
\setlength{\parskip}{6pt plus 2pt minus 1pt}
}
\setlength{\emergencystretch}{3em}  % prevent overfull lines
\providecommand{\tightlist}{%
  \setlength{\itemsep}{0pt}\setlength{\parskip}{0pt}}
\setcounter{secnumdepth}{5}
% Redefines (sub)paragraphs to behave more like sections
\ifx\paragraph\undefined\else
\let\oldparagraph\paragraph
\renewcommand{\paragraph}[1]{\oldparagraph{#1}\mbox{}}
\fi
\ifx\subparagraph\undefined\else
\let\oldsubparagraph\subparagraph
\renewcommand{\subparagraph}[1]{\oldsubparagraph{#1}\mbox{}}
\fi

%%% Use protect on footnotes to avoid problems with footnotes in titles
\let\rmarkdownfootnote\footnote%
\def\footnote{\protect\rmarkdownfootnote}

%%% Change title format to be more compact
\usepackage{titling}

% Create subtitle command for use in maketitle
\newcommand{\subtitle}[1]{
  \posttitle{
    \begin{center}\large#1\end{center}
    }
}

\setlength{\droptitle}{-2em}
  \title{Exploring data using R}
  \pretitle{\vspace{\droptitle}\centering\huge}
  \posttitle{\par}
  \author{Kamarul Imran Musa, Wan Nor Arifin}
  \preauthor{\centering\large\emph}
  \postauthor{\par}
  \predate{\centering\large\emph}
  \postdate{\par}
  \date{2017-10-27}


\usepackage{amsthm}
\newtheorem{theorem}{Theorem}[chapter]
\newtheorem{lemma}{Lemma}[chapter]
\theoremstyle{definition}
\newtheorem{definition}{Definition}[chapter]
\newtheorem{corollary}{Corollary}[chapter]
\newtheorem{proposition}{Proposition}[chapter]
\theoremstyle{definition}
\newtheorem{example}{Example}[chapter]
\theoremstyle{definition}
\newtheorem{exercise}{Exercise}[chapter]
\theoremstyle{remark}
\newtheorem*{remark}{Remark}
\newtheorem*{solution}{Solution}
\begin{document}
\maketitle

{
\setcounter{tocdepth}{1}
\tableofcontents
}
\chapter{Introduction to R}\label{introduction-to-r}

\section{R and RStudio}\label{r-and-rstudio}

\subsection{Installation of R}\label{installation-of-r}

\subsection{Starting R}\label{starting-r}

\subsection{Installation of RStudio}\label{installation-of-rstudio}

\subsubsection{Starting RStudio}\label{starting-rstudio}

\subsubsection{Why RStudio?}\label{why-rstudio}

\subsubsection{RStudio interface}\label{rstudio-interface}

\section{Working with packages}\label{working-with-packages}

\subsection{About packages}\label{about-packages}

\subsection{Package installation}\label{package-installation}

\subsection{Loading packages}\label{loading-packages}

\section{Working directory}\label{working-directory}

\subsection{Setting a working
directory}\label{setting-a-working-directory}

\section{Data management}\label{data-management}

\subsection{Reading data set}\label{reading-data-set}

\subsection{Viewing data set}\label{viewing-data-set}

\subsection{Exporting data set from R}\label{exporting-data-set-from-r}

\section{More about data management}\label{more-about-data-management}

\subsection{Subsetting}\label{subsetting}

\subsubsection{Selecting a column (variable) or a row
(observation)}\label{selecting-a-column-variable-or-a-row-observation}

\subsubsection{Selecting columns}\label{selecting-columns}

\subsubsection{Selecting rows}\label{selecting-rows}

\subsubsection{Select rows and columns
together}\label{select-rows-and-columns-together}

\subsubsection{Creating a new variable}\label{creating-a-new-variable}

\subsection{Recoding}\label{recoding}

\subsection{Categorize into new
variables}\label{categorize-into-new-variables}

\subsubsection{From a numerical
variable}\label{from-a-numerical-variable}

\subsubsection{From a categorical
variable}\label{from-a-categorical-variable}

\subsection{Direct data entry}\label{direct-data-entry}

\section{Summary}\label{summary}

\chapter{Textual}\label{textual}

\section{Basic descriptive
statistics}\label{basic-descriptive-statistics}

\section{\texorpdfstring{Data sets in package
`datasets':}{Data sets in package datasets:}}\label{data-sets-in-package-datasets}

\section{AirPassengers Monthly Airline Passenger Numbers
1949-1960}\label{airpassengers-monthly-airline-passenger-numbers-1949-1960}

\section{BJsales Sales Data with Leading
Indicator}\label{bjsales-sales-data-with-leading-indicator}

\section{BJsales.lead (BJsales) Sales Data with Leading
Indicator}\label{bjsales.lead-bjsales-sales-data-with-leading-indicator}

\section{BOD Biochemical Oxygen
Demand}\label{bod-biochemical-oxygen-demand}

\section{CO2 Carbon Dioxide Uptake in Grass
Plants}\label{co2-carbon-dioxide-uptake-in-grass-plants}

\section{\ldots{}}\label{section}

\subsection{Describing a numerical
variable}\label{describing-a-numerical-variable}

\subsection{Describing a categorical
variable}\label{describing-a-categorical-variable}

\section{More on descriptive
statistics}\label{more-on-descriptive-statistics}

\subsection{Describing numerical
variables}\label{describing-numerical-variables}

\subsection{Describing categorical
variables}\label{describing-categorical-variables}

\subsection{Describing the variables
together}\label{describing-the-variables-together}

\subsubsection{By groups}\label{by-groups}

\subsubsection{Simple cross-tabulation}\label{simple-cross-tabulation}

\section{Summary}\label{summary-1}

\chapter{Graphical}\label{graphical}

\section{Preliminaries}\label{preliminaries}

\subsection{Reading dataset}\label{reading-dataset}

\subsection{Describing data}\label{describing-data}

\section{One variable}\label{one-variable}

\subsection{One variable: A categorical or factor
variable}\label{one-variable-a-categorical-or-factor-variable}

\subsection{One variable: A numerical
variable}\label{one-variable-a-numerical-variable}

\section{Two variables}\label{two-variables}

\subsection{Two variables : A numerical with another numerical
variable}\label{two-variables-a-numerical-with-another-numerical-variable}

\subsection{Two variables : A categorical variable with a categorical
variable}\label{two-variables-a-categorical-variable-with-a-categorical-variable}

\section{Summary}\label{summary-2}

\chapter{Reporting results}\label{reporting-results}

\chapter{Summary}\label{summary-3}

\section{What we have learned so far}\label{what-we-have-learned-so-far}

\section{Some important packages}\label{some-important-packages}

\chapter{References}\label{references}

\chapter{Preparation}\label{preparation}

\section{Set the working directory}\label{set-the-working-directory}

\section{Read data}\label{read-data}

\section{Training data}\label{training-data}

\section{Questions to ask when graphing or plotting
data}\label{questions-to-ask-when-graphing-or-plotting-data}

\section{One variable: A categorical
variable}\label{one-variable-a-categorical-variable}

\section{One variable: A numerical
variable}\label{one-variable-a-numerical-variable-1}

\subsection{Histogram}\label{histogram}

\subsection{Kernel density plot}\label{kernel-density-plot}

\section{Two variables : A numerical with another numerical
variable}\label{two-variables-a-numerical-with-another-numerical-variable-1}

\subsection{Scatterplot}\label{scatterplot}

\section{One variable: A numerical
variable}\label{one-variable-a-numerical-variable-2}

\subsection{Histogram}\label{histogram-1}

\subsection{Density curve}\label{density-curve}

\subsection{Combining the histogram and the density
curve}\label{combining-the-histogram-and-the-density-curve}

\section{One variable: A categorical
variable}\label{one-variable-a-categorical-variable-1}

\section{Two variables: A numerical and a categorical
variable}\label{two-variables-a-numerical-and-a-categorical-variable}

\subsection{Overlaying histograms}\label{overlaying-histograms}

\subsection{Interleaving histograms}\label{interleaving-histograms}

\subsection{Overlaying density plots}\label{overlaying-density-plots}

\subsection{Using facets}\label{using-facets}

\section{Two or more categorical
variables}\label{two-or-more-categorical-variables}

\section{Data and package}\label{data-and-package}

\subsection{Package}\label{package}

\subsection{Training data}\label{training-data-1}

\section{View data}\label{view-data}

\section{One numerical variable -
Histogram}\label{one-numerical-variable---histogram}

\section{One numerical and one categorical variable -
histogram}\label{one-numerical-and-one-categorical-variable---histogram}

\section{Barchart}\label{barchart}

\section{Scatterplot}\label{scatterplot-1}

\section{Reading dataset from STATA file
(.dta)}\label{reading-dataset-from-stata-file-.dta}

\section{Graphing or Plotting data}\label{graphing-or-plotting-data}

\subsection{One variable: A categorical or factor
variable}\label{one-variable-a-categorical-or-factor-variable-1}

\subsection{One variable: A numerical
variable}\label{one-variable-a-numerical-variable-3}

\subsection{Two variables : a numerical with a categorical
variable}\label{two-variables-a-numerical-with-a-categorical-variable}

\section{Saving plot}\label{saving-plot}

\chapter{Data transformation (data munging or data
wrangling)}\label{data-transformation-data-munging-or-data-wrangling}

\section{Definition of data
wrangling}\label{definition-of-data-wrangling}

\section{Package: dplyr}\label{package-dplyr}

\section{Data wrangling using dplyr}\label{data-wrangling-using-dplyr}

\section{Preparation and data}\label{preparation-and-data}

\subsection{Working directory and data
format}\label{working-directory-and-data-format}

\subsection{Training data}\label{training-data-2}

\section{Function: select and mutate}\label{function-select-and-mutate}

\subsection{Select}\label{select}

\subsection{Mutate}\label{mutate}

\section{arrange and filter}\label{arrange-and-filter}

\subsection{arrange}\label{arrange}

\subsection{filter}\label{filter}

\section{group\_by}\label{group_by}

\subsection{Summarize data -
summarize}\label{summarize-data---summarize}

\section{Summary}\label{summary-4}

\section{Categorical variables}\label{categorical-variables}

\section{forcats}\label{forcats}

\subsection{New dataset}\label{new-dataset}

\subsection{Convert numeric to factor
variables}\label{convert-numeric-to-factor-variables}

\subsection{Recode old to new levels}\label{recode-old-to-new-levels}


\end{document}
